\section{Introduction}

Programming languages with dynamic typing, such as Python or JavaScript, are increasingly popular. In fact, supported by the increasing use of machine learning, Python is currently the top programming language in the IEEE Spectrum rankings \cite{2019Interactive:Languages}. Dynamically typed languages do not require manual type annotations and only know the types of variables at run-time. They provide much flexibility and are therefore very suitable for beginners and for fast prototyping. There are, however, drawbacks when a project grows large enough that no single developer knows every single code element of the project. At that point, statically typed languages can check certain naming behaviors based on types automatically whereas dynamic typing requires manual intervention. While there is an ongoing debate on static vs. dynamic typing in the developer community \cite{Ray2017AGitHub}, both excel in certain aspects \cite{Meijer2004StaticLanguages}. There is scientific evidence that static typing provides certain benefits that are useful when software needs to be optimized for efficiency, modularity or safety \cite{Vitousek2014DesignPython}. The benefits include better auto-completion in integrated development environments (IDEs) \cite{Malik2019NL2Type:Information}, more efficient code generation \cite{Vitousek2014DesignPython}, improved maintainability \cite{Hanenberg2014AnMaintainability}, better readability of undocumented source code \cite{Hanenberg2014AnMaintainability}, and preventing certain run-time crashes \cite{Malik2019NL2Type:Information}. It has also been shown that statically typed languages are "less error-prone than functional dynamic languages" \cite{Ray2017AGitHub}.

Weakly typed languages, such as JavaScript, PHP or Python do not provide these benefits. When static typing is needed, there are usually two solutions available. Either using optional type syntax within the language or to use a variant of the language, which is essentially a different language, that does have a type system in place. For JavaScript, there are two often used solutions: Flow \cite{2014Flow}, which uses type annotations within JavaScript, and TypeScript \cite{2012TypeScript}, a JavaScript variant. PHP offers support for type declarations since PHP 7.0 \footnote{https://www.php.net/manual/en/migration70.new-features.php}, and checks these types at run-time. A well-known PHP variant that has strong typing is HackLang \cite{2014HackLang}, by Facebook. Python has support for typing since Python 3.5 \footnote{https://www.python.org/dev/peps/pep-0484/}. It does not do any run-time checking, and therefore these types do not provide any guarantees if no type checker is run. The type checker mypy \cite{2012Mypy} is the most used Python type checker.

Type inference for Python has been addressed from multiple angles \cite{Xu2016PythonSupport,Salib2004FasterStarkiller,MacLachlan1992TheLisp,Hassan2018MaxSMT-Based3c,Maia2012APython}. However, these solutions require some manual annotations to provide accurate results. Having to provide manual annotations is one of the main arguments against static typing because this lowers developer productivity.

In an attempt to mitigate this need for manual annotations and support developers in typing their codebases, we present \dltpy: a deep learning type inference solution based on natural language for the prediction of Python function types. Our work focuses on answering the question of how effective this approach is.

\dltpy{} follows the ideas behind NL2Type \cite{Malik2019NL2Type:Information}, a similar learning-based approach for JavaScript function types. Our solution makes predictions based on comments, on the semantic elements of the function name and argument names, and on the semantic elements of identifiers in the return expressions. The latter is an extension of the ideas proposed in \cite{Malik2019NL2Type:Information}. The idea to use natural language contained in the parameter names for type predictions in Python is not new, Zhaogui Xu et al. already used this idea to develop a probabilistic type inferencer \cite{Xu2016PythonSupport}. Using the natural language of these different elements, we can train a classifier that predicts types. Similar to \cite{Malik2019NL2Type:Information} we use a recurrent neural network (RNN) with a Long Short-Term Memory (LSTM) architecture \cite{Hochreiter1997LongMemory}.

Using 5,996 open source projects mined from GitHub and Libraries.io that are likely to have type annotations, we train the model to predict types of functions without annotations. This works because code has been shown to be repetitive and predictable \cite{Hindle2012OnSoftware}. We make the assumption that comments and identifiers convey the intent of a function \cite{Malik2019NL2Type:Information}.

We train and test multiple variants of \dltpy{} to evaluate the usefulness of certain input elements and the success of different deep learning models. We find that return expressions are improving the accuracy of the model, and that including comments has a positive influence on the results. 

\begin{notsw}
\dltpy{} predicts types with a top-3 precision of 91.4\%, a top-3 recall of 91.9\%, and a top-3 F1-score of 91.6\%. \dltpy{} does not significantly outperform or underperform the previous work NL2Type \cite{Malik2019NL2Type:Information}.
\end{notsw}

This paper's contributions are three-fold:
\begin{enumerate}
    \item A deep learning network type inference system for inferring types of Python functions
    \item Evaluation of the usefulness of natural language encoded in return expressions for type predictions
    \item Evaluation of different deep learning models that indicates their usefulness for this classification task
\end{enumerate}

% * Previous work in finding types for python (e.g. static analysis, check [4-7] of nl2type)
% * We present... Look at natural langauge of the function context (comments, params etc.). We formulate the task as a clasification problem. 
%     * Cite the four reasosns why this works
%     * Online learning
% * More developers are using Python, also used in some major packages such as .... We use this data to learn types blabla.
% * Somethin about the results