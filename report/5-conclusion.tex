\section{Conclusion}
% objective / research question
Our work set out to study the applicability of using natural language to infer types of Python function parameters and return values.
We present \dltpy{} as a deep learning type inference solution based on natural language for the prediction of these types.
It uses information from function names, comments, parameter names, and return expressions. 

% answer
The results show that \dltpy{} is effective at predicting types using natural language information. The top-1 F1-score is 82.4\%, and the top-3 F1-score is 91.6\%. This shows that in most of the cases the correct answer is in the top 3 of predictions. The results show that using natural language information from the context of a function and using return expressions have a positive impact on the results of the type prediction task.

We do not significantly outperform or underperform NL2Type. Without our additions to the ideas behind NL2Type, however, \dltpy{} would underperform. This shows that the main idea behind NL2Type, namely using natural language information for predicting types, is generalizable from JavaScript to Python, but additional information, such as return expressions, is needed to get comparable results.

We identify two threats to the validity of our results. The first threat is that there is no separation of functions between the training and test set on project level. Because functions within the same project are more likely to be similar, this might influence the validity of our results. Also, since the best performing model, model C, has 404,456 parameters and the best performing dataset, dataset 1, has just 84,661 datapoints, which increases the risk of overfitting.

\dltpy{} has limitations that can be improved upon in future work. Dataset 1, the complete dataset, is relatively small. A better data retrieval strategy that goes beyond looking at mypy dependents might result in more data points and thus allows for better training resulting in more accurate results. Furthermore, the predictions of \dltpy{} are currently restricted to the 1000 most frequently used types in Python. Open type predictions would improve the practical use of \dltpy{}, given that there is enough training data available for the types that are less frequent.
